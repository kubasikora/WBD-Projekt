\documentclass{mwrep}

% Polskie znaki
\usepackage{polski}
\usepackage[utf8]{inputenc}
\usepackage[T1]{fontenc}
\usepackage{lmodern}
\usepackage{indentfirst}

% Strona tytułowa
\usepackage{pgfplots}
\usepackage{siunitx}
\usepackage{paracol}

% Pływające obrazki
\usepackage{float}
\usepackage{svg}
\usepackage{graphicx}

% table of contents refs
\usepackage{hyperref}
\usepackage{cleveref}
\usepackage{booktabs}
\usepackage{listings}



\SendSettingsToPgf
\title{\bf System biurowych zakładów wzajemnych \vskip 0.1cm}
\author{Robert Wojtaś \and Jakub Sikora}
\date{\today}
\pgfplotsset{compat=1.15}	
\begin{document}

\makeatletter
\renewcommand{\maketitle}{\begin{titlepage}
		\begin{center}{
				\LARGE {\bf Politechnika Warszawska}}\\
			\vspace{0.4cm}
			{\LARGE {\bf Wydział Elektroniki i Technik Informacyjnych}}\\
			\vspace{5cm}
			{\bf \LARGE \mbox{Wprowadzenie do Baz Danych - Projekt} \vskip 0.1cm}
		\end{center}
		\vspace{0.1cm}

		\begin{center}
			{\bf \LARGE \@title}
		\end{center}

		\vspace{10cm}
		\begin{paracol}{2}
			\addtocontents{toc}{\protect\setcounter{tocdepth}{1}}
			\subsection*{Zdający:}
			\bf{ \Large{ \noindent\@author \par}}
			\addtocontents{toc}{\protect\setcounter{tocdepth}{2}}

			\switchcolumn \addtocontents{toc}{\protect\setcounter{tocdepth}{1}}
			\subsection*{Prowadzący:}
			\bf{\Large{\noindent \mbox{dr inż. Marcin Kowalczyk}}}
			\addtocontents{toc}{\protect\setcounter{tocdepth}{2}}

		\end{paracol}
		\vspace*{\stretch{6}}
		\begin{center}
			\bf{\large{Warszawa, \@date\vskip 0.1cm}}
		\end{center}
	\end{titlepage}
}
\makeatother
\maketitle

\tableofcontents

\chapter{Zakres i cel projektu}


\chapter{Definicja systemu}

\section{Perspektywy użytkowników}


\chapter{Model konceptualny}

\section{Definicja zbiorów encji określonych w projekcie}

\section{Ustalenie związków i ich typów między encjami}

\section{Określenie atrybutów i ich dziedzin}

\section{Dodatkowe reguły integralnościowe}

\section{Klucze kandydujące i główne}

\section{Schemat ER na poziomie konceptualnym}

\section{Problem pułapek szczelinowych i wachlarzowych – analiza i przykłady}



\chapter{Model logiczny}

\section{Charakterystyka modelu relacyjnego}

\section{Usunięcie właściwości niekompatybilnych z modelem relacyjnym}

\section{Proces normalizacji}

\subsection{Pierwsza postać normalna}
\subsection{Drugaa postać normalna}
\subsection{Trzecia postać normalna}


\section{Schemat ER na poziomie modelu logicznego}

\section{Więzy integralności}

\section{Proces denormalizacji}

\chapter{Faza fizyczna}

\section{Projekt transakcji i weryfikacja ich wykonalności}

\section{Strojenie bazy danych – dobór indeksów}

\section{Skrypt SQL zakładający bazę danych}

\section{Przykłady zapytań i poleceń SQL odnoszących się do bazy danych}

\end{document}