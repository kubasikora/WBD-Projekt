\documentclass{mwrep}[15pt]

% Polskie znaki
\usepackage{polski}
\usepackage[utf8]{inputenc}
\usepackage[T1]{fontenc}
\usepackage{lmodern}
\usepackage{indentfirst}

% Strona tytułowa
\usepackage{pgfplots}
\usepackage{siunitx}
\usepackage{paracol}

% Pływające obrazki
\usepackage{float}
\usepackage{svg}
\usepackage{graphicx}

% table of contents refs
\usepackage{hyperref}
\usepackage{cleveref}
\usepackage{booktabs}
\usepackage{listings}
\usepackage{threeparttable}


\SendSettingsToPgf
\title{\bf System biurowych zakładów wzajemnych \vskip 0.1cm}
\author{Robert Wojtaś \and Jakub Sikora}
\date{\today}
\pgfplotsset{compat=1.15}	
\begin{document}

\makeatletter
\renewcommand{\maketitle}{\begin{titlepage}
		\begin{center}{
				\LARGE {\bf Politechnika Warszawska}}\\
			\vspace{0.4cm}
			{\LARGE {\bf Wydział Elektroniki i Technik Informacyjnych}}\\
			\vspace{5cm}
			{\bf \LARGE \mbox{Wprowadzenie do Baz Danych - Projekt} \vskip 0.1cm}
		\end{center}
		\vspace{0.1cm}

		\begin{center}
			{\bf \LARGE \@title}
		\end{center}

		\vspace{10cm}
		\begin{paracol}{2}
			\addtocontents{toc}{\protect\setcounter{tocdepth}{1}}
			\subsection*{Zdający:}
			\bf{ \Large{ \noindent\@author \par}}
			\addtocontents{toc}{\protect\setcounter{tocdepth}{2}}

			\switchcolumn \addtocontents{toc}{\protect\setcounter{tocdepth}{1}}
			\subsection*{Prowadzący:}
			\bf{\Large{\noindent \mbox{dr inż. Marcin Kowalczyk}}}
			\addtocontents{toc}{\protect\setcounter{tocdepth}{2}}

		\end{paracol}
		\vspace*{\stretch{6}}
		\begin{center}
			\bf{\large{Warszawa, \@date\vskip 0.1cm}}
		\end{center}
	\end{titlepage}
}
\makeatother
\maketitle

\tableofcontents

\chapter{Zakres i cel projektu}

\section{Cel projektu}
Celem projektu jest poprawne zaprojektowanie relacyjnej bazy danych oraz jej
fizyczna implementacja przy użyciu systemu Oracle. W trakcie projektowania należy
odpowiednio podzielić fazy projektowania na poziom konceptualny oraz logiczny a także 
doprowadzić projekt bazy do trzeciej postaci normalnej.

\section{Założenia projektowe}
Realizowany projekt dotyczy biurowego systemu obstawiania meczów
na dużych turniejach sportowych. System ten zajmuje się zbieraniem 
zakładów od swoich użytkowników oraz prowadzeniem statystyk. 
Oferuje swoim użytkownikom możliwość zakładania się na wyniki spotkań
rozgrywanych w ramach uprzednio zdefiniowanych turniejów oraz podliczaniem 
wyników wedle ustalonego algorytmu punktowania.
\\ 
\\
\indent W tym celu, system prowadzi bazę danych która zbiera informacje o spotkaniach
oraz dostępnych turniejach. Każdy uczestnik gry próbuje przewidzieć dokładny wynik 
spotkania i w zależności od poprawności, otrzymuje $3$, $1$ albo $0$ punktów.
Trzy punkty gracz otrzymuje, gdy padnie dokładnie obstawiony przez niego wynik.
Gracz otrzymuje jeden punkt, gdy końcowy rezultat jest taki sam jak obstawiony z dokładnością
do zdobytych przez drużynę punktów. Przykładowo, gracz obstawił że meczu piłki
nożnej zakończy się rezultatem $3$:$1$ a mecz zakończył się wynikiem $4$:$2$. 
Gdy gracz nie trafi w rezultat, nie otrzymuje punktów.  
\\
\\
\indent W celu ułatwienia graczom podejmowanie decyzji, system oferuje szeroką gamę statystyk 
prowadzonych w ramach turniejów. Baza przechowuje informacje o nadchodzących spotkaniach 
pomiędzy dwoma drużynami, o zawodnikach występujących w tych drużynach, o sędziach 
przewidzianych do prowadzania danego spotkania a także o planowanym miejscu rozegrania spotkania.  
\\
\\
\indent Każdy użytkownik może grać niezależnie w kilku różnych turniejach i dodawać zakłady 
na dowolną ilość spotkań z takim zastrzeżeniem, że nie wolno dodać zakładu na 
rozpoczęte już spotkania. Dodatkowo, każdy turniej oferuje specjalne zakłady długoterminowe
dotyczące ostatecznego zwycięzcy i najbardziej wartościowego zawodnika turnieju. Zakłady 
te są warte odpowiednio więcej punktów. Po zakończeniu wszystkich spotkań z danego turnieju, wybierany 
jest zwycięzca na podstawie zdobytej liczby punktów. 


\chapter{Definicja systemu}

\section{Perspektywy użytkowników}
W ramach systemu zdefiniowaliśmy trzy typy potencjalnych użytkowników: \\
\begin{enumerate}
    \item Użytkownik - uczestnik gry, dodaje zakłady nierozpoczęte jeszcze spotkania w
    ramach turniejów na które się wcześniej zapisał\\ 

    \item Statystyk - moderator gry, dodaje informacje o drużynach, sędziach, stadionach i trenerach
    oraz na bieżąco uzupełnia wyniki spotkań i turniejów \\ 

    \item Administrator - główny moderator systemu, tworzy konta użytkownikom oraz przywraca dostępy \\
    
\end{enumerate}

\section{Zdefiniowane funkcjonalności}
Do zdefiniowania funkcjonalności posłużyliśmy się metodyką User Stories znanych z metodyki Agile. Wcieliśmy się 
w rolę każdego z użytkowników i opisaliśmy potrzebne funkcjonalności według znanego schematu:

\begin{center}
    \emph{Jako} osoba ..., \\ \emph{potrzebuję/chcę} takiej funkcjonalności ...,\\  \emph{ponieważ} pozwoli mi to ...  
\end{center}

Na podstawie tak opisanych funkcjonalności, w łatwy sposób mogliśmy zdefiniować potrzebne transakcje w systemie. Dodatkowo,
taki sposób opisu funkcjonalności już na etapie projektowania sprawdza ich przydatność.

\subsection{Użytkownik systemu}

\subsubsection{Zapis na turniej}
\emph{Jako użytkownik, potrzebuję mieć możliwość zapisania się na turniej, ponieważ wtedy będę mógł wziąć udział w grze.}

\subsubsection{Dodawanie zakładu na spotkanie}
\emph{Jako użytkownik, potrzebuję móc dodawać nowe zakłady na spotkania które jeszcze się nie rozpoczęły, ponieważ pozwoli mi to na zdobywanie punktów.}

\subsubsection{Edycja zakładu na spotkanie}
\emph{Jako użytkownik, potrzebuję móc edytować swoje zakłady na spotkania które jeszcze się nie rozpoczęły, ponieważ pozwoli mi to na zmianę zdania i poprawienie swojego zakładu.}

\subsubsection{Dodawanie zakładu długoterminowego}
\emph{Jako użytkownik, potrzebuję móc dodawać zakład długoterminowy na zwycięzce turnieju na który się zapisałem a który jeszcze się nie rozpoczął, ponieważ dzięki temu będę mógł zdobyć więcej punktów.}

\subsubsection{Edycja zakładu długoterminowego}
\emph{Jako użytkownik, potrzebuję móc edytować zakład długoterminowego na zwycięzce turnieju na który się zapisałem a który jeszcze się nie rozpoczął, ponieważ pozwoli mi to na zmianę zdania i poprawienie swojego zakładu.}

\subsubsection{Podgląd tabeli}
\emph{Jako użytkownik, potrzebuję móc sprawdzać który jestem w tabeli wyników, ponieważ potrzebuję informacji o ewentualnym zwycięstwie.}

\subsection{Statystyk}
\subsubsection{Dodawanie turniejów}
\emph{Jako statystyk, potrzebuję możliwości dodawania turniejów, ponieważ wtedy użytkownicy będą mogli się na nie zapisywać.}

\subsubsection{Dodawanie meczów}
\emph{Jako statystyk, potrzebuję możliwości dodawania meczów w ramach turniejów, ponieważ wtedy użytkownicy będą mogli się próbować obstawić ich wynik.}

\subsubsection{Dodawanie drużyn}
\emph{Jako statystyk, potrzebuję możliwości dodawania drużyn do systemu, ponieważ wtedy będę mógł je podpinać pod turnieje.}

\subsubsection{Podpinanie drużyn do turniejów}
\emph{Jako statystyk, potrzebuję możliwości dodawania drużyn jako uczestników turniejów, ponieważ wtedy będę mógł dodawać mecze do turniejów pomiędzy nimi.}

\subsubsection{Dodawanie stadionów}
\emph{Jako statystyk, potrzebuję możliwości dodawania stadionów do systemu, ponieważ wtedy będę mógł podpinać stadiony do meczów.}

\subsubsection{Podpinanie stadionów}
\emph{Jako statystyk, potrzebuję możliwości podpinania stadionów do meczów, ponieważ wtedy użytkownicy będą więcej wiedzieli o meczu co pozwoli im lepiej obstawić.}

\subsubsection{Dodawanie sędziów}
\emph{Jako statystyk, potrzebuję możliwości dodawania sędziów, ponieważ wtedy będę mógł podpinać sędziów głównych do spotkań.}

\subsubsection{Podpinanie sędziów}
\emph{Jako statystyk, potrzebuję możliwości podpinania sędziów do meczów, ponieważ wtedy użytkownicy będą więcej wiedzieli o meczu co pozwoli im lepiej obstawić.}

\subsubsection{Dodawanie wyników meczów}
\emph{Jako statystyk, potrzebuję możliwości dodawania wyników zakończonych już spotkań, ponieważ wtedy użytkownicy będą mogli zweryfikować swoje zakłady.}

\subsubsection{Dodawania zwycięzców turniejów}
\emph{Jako statystyk, potrzebuję możliwości dodawania zwycięzców turniejów, ponieważ wtedy użytkownicy będą mogli zweryfikować swoje zakłady długoterminowe.}


\subsection{Administrator}
\subsubsection{Dodawanie użytkowników}
\emph{Jako administrator, potrzebuję możliwości tworzenia kont dla użytkowników, ponieważ wtedy będą mogli wziąc udział w grze.}

\subsubsection{Zmiana hasła użytkownika}
\emph{Jako administrator, potrzebuję możliwości zmiany hasła użytkownika, ponieważ wtedy będę mógł przywrócić dostęp tym którzy zapomnieli hasła.}


\chapter{Model konceptualny}
Modelem konceptualnym nazywamy pewną reprezentację obiektów świata rzeczywistego 
w uniwersalnym modelu niezależnym od implementacji \cite{Wrembel1}. Poprawny model 
konceptualny jest dobrym punktem wyjścia, ponieważ pozwala na przemyślenie jakie
dane tak naprawdę chcemy przechowywać w projektowanej bazie oraz pozwala na usystematyzowanie
związków między danymi.

\section{Definicja zbiorów encji określonych w projekcie}
Wymagany zbiór encji najłatwiej uzyskać poprzez wyodrębnienie rzeczowników z opisu projektu.
Na podstawie analizy założeń projektowych założyliśmy następujący zbiór encji:

\begin{itemize}
	\item Użytkownik
	\item Mecz
	\item Zakład
	\item Wynik
	\item Zakład długoterminowy
	\item Turniej
	\item Drużyna
	\item Stadion
	\item Trener
	\item Sędzia
\end{itemize}
 
\vspace{1cm}
\section{Ustalenie związków i ich typów między encjami}

\begin{threeparttable}[H]
	\begin{tabular}{|p{0.15\linewidth}|p{0.15\linewidth}|p{0.05\linewidth}|p{0.17\linewidth}|p{0.17\linewidth}|p{0.1\linewidth}|}
	\hline
	\multicolumn{2}{|p{0.3\linewidth}|}{Relacje biorące udział} & Typ & \multicolumn{2}{p{0.3\linewidth}|}{Obowiązkowość} & Stopień         \\ \hline
	Użytkownik & Zakład & 1:n & Obowiązkowy & Opcjonalny & binarny  \\ \hline
	Użytkownik & ZakładD\tnote{1} & 1:n & Obowiązkowy & Opcjonalny & binarny  \\ \hline
	Użytkownik & Turniej & n:m & Opcjonalny & Opcjonalny & binarny  \\ \hline
	Mecz & Zakład & 1:n & Obowiązkowy & Opcjonalny & binarny  \\ \hline
	Turniej & ZakładD\tnote{1} & 1:n & Obowiązkowy & Opcjonalny & binarny  \\ \hline
	Turniej & Mecz & 1:n & Obowiązkowy & Opcjonalny & binarny  \\ \hline
	Sędzia & Mecz & 1:n & Obowiązkowy & Opcjonalny & binarny  \\ \hline
	Stadion & Mecz & 1:n & Obowiązkowy & Opcjonalny & binarny  \\ \hline
	Wynik & Mecz & 1:1 & Opcjonalny & Obowiązkowy & binarny  \\ \hline
	Drużyna & ZakładD\tnote{1} & 1:1 & Obowiązkowy & Obowiązkowy & binarny \\ \hline
	Drużyna & Mecz & 2:n & Obowiązkowy & Opcjonalny & binarny  \\ \hline
	Turniej & Drużyna & n:m & Opcjonalny & Opcjonalny & binarny \\ \hline 
	Turniej & Drużyna & 1:n & Opcjonalny & Opcjonalny & binarny \\ \hline 
	Zawodnik & Drużyna & n:m & Obowiązkowy & Opcjonalny & binarny  \\ \hline
	Trener & Drużyna & n:m & Opcjonalny & Obowiązkowy & binarny  \\ \hline
	\end{tabular}
	\begin{tablenotes}
		\item [1] skrót od ZakładDługoterminowy
	\end{tablenotes}	
	\caption{Tabela typów związków}
\end{threeparttable}

\subsubsection{Użytkownik - Zakład}
Związek użytkownika z zakładem odwzorowuje następującą zależność. Użytkownik stawia zakład. Każdy zakład ma swojego jednego właściciela którym jest dany użytkownik, natomiast jeden użytkownik może stawiać od zera do wielu zakładów.

\subsubsection{Użytkownik - ZakładDługoterminowy}
Związek użytkownika z zakładem długoterminowym jest przedstawieniem w systemie zdarzenia postawienia przez użytkownika zakładu długoterminowego. Użytkownik tworzy zakład długoterminowy,w ogólności może ich tworzyć wiele (może też nie stworzyć żadnego). Każdy zakład długoterminowy ma obowiązkowo swojego jedynego właściciela.

\subsubsection{Użytkownik - Turniej}
Związek pomiędzy użytkownikiem a turniejem symbolizuje zapis użytkownika na turniej. Na jeden turniej może być zapisanych od zera do wielu użytkowników. Każdy użytkownik może zapisać się od zera do wielu turniejów.

\subsubsection{Mecz - Zakład} 
Kolejny związek jest reprezentacją fizycznego połączenia meczu i zakładu. Jeden zakład przewiduje wynik
tylko jednego meczu. Na jeden mecz może być postawionych kilka zakładów, w szczególności zero.

\subsubsection{Turniej - ZakładDługoterminowy}
Każdy zakład długoterminowy przewiduje wynik dokładnie jednego turnieju. Na jeden turniej może zostać
postawionych od zera do wielu zakładów długoterminowych.

\subsubsection{Turniej - Mecz}
W ramach jednego turnieju rozgrywane jest od zera (sytuacja raczej dziwna, jednak możliwa zanim statystyk doda do turnieju drużyny oraz spotkania) do wielu
spotkań. Jeden mecz jest rozgrywany w ramach jednego turnieju, przy czym zgodnie z założeniem wykluczamy możliwość
obstawiania spotkań towarzyskich czyli takich które nie są rozgrywane w ramach żadnego turnieju.

\subsubsection{Sędzia - Mecz}
Reprezentuje sędziowanie meczu przez sędziego. Jeden sędzia może sędziować wiele spotkań (oczywiście nie równocześnie), natomiast 
jedno spotkanie ma tylko jednego sędziego głównego.

\subsubsection{Stadion - Mecz}
Podobnie jak w poprzednim związku, jeden mecz jest rozgrywany na jednym stadionie. Jeden stadion może być gospodarzem
wielu spotkań.

\subsubsection{Wynik - Mecz}
Każdy mecz kończy się jakimś wynikiem. Zdecydowaliśmy się na rozwiązanie w którym wynik meczu jest przechowywany w osobnej encji.
Jest to związek 1:1, jeden wynik obowiązkowo odpowiada jednemu spotkaniu. Warto zwrócić uwagę na opcjonalność związku ze strony encji meczu.
Mecz który się jeszcze nie zakończył, nie może posiadać wyniku (system nie przewiduje obstawiania ustawionych spotkań).

\subsubsection{Drużyna - Zakład Długoterminowy}
Reprezentuje przewidywany wynik zakładu długoterminowego. Jeden zakład długoterminowy przewiduje jednego zwycięzce danego turnieju. 
Dana drużyna może wielokrotnie zostać wybrana przez użytkownika jako potencjalny zwycięzca, choć wcale nie musi być wybrana kiedykolwiek.

\subsubsection{Drużyna - Mecz}
Związek reprezentuje uczestnictwo drużyn w meczu. Każdy mecz musi mieć swoich uczestników. Zgodnie z założeniem, system pozwala na 
obstawianie spotkań pomiędzy dwoma drużynami, nie będą obsługiwane dyscyplinu typu skoki narciarskie gdzie liczba uczestników jest różna od dwóch.
Jedna drużyna może grać wiele spotkań, w szczególności żadnego.


\subsubsection{Turniej- Drużyna }
W ramach jednego turnieju, gra w nim tylko określony zbiór drużyn. Zgodnie z założeniem, drużyny mogą być dynamicznie dodawane 
do turnieju dlatego też zakładamy że turniej może mieć zero uczestniczących drużyn. Z drugiej strony, jedna drużyna może brać udział 
w wielu turniejach czy chociażby w różnych edycjach tych samych rozgrywek.


\subsubsection{Turniej - Drużyna \emph{ponownie...}}
Turnieje i drużyny łączy jeszcze jeden związek, który zdecydowaliśmy się specjalnie wyróżnić. Drugi związek pomiędzy turniejem a drużyną, reprezentuje zwycięzcę turnieju. W przypadku zwycięzców turniejów, 
zdecydowaliśmy się na przedstawienie tego nie za pomocą kolejnej encji (tak jak w przypadku encji \emph{Wynik}), tylko 
przy pomocy związku. Jeden turniej ma jednego zwycięzce, który zostaje poznany po jego zakończeniu. Jedna drużyna może wygrywać kilka turniejów,
w szczególności nie musi wygrać jakiegokolwiek.


\subsubsection{Zawodnik - Drużyna}
Jedna drużyna w danym momencie może mieć zakontraktowanych wielu zawodników oraz dodatkowo mieć historię zakończonych kontraktów z 
jeszcze większą ilością sportowców. Z drugiej strony, jeden zawodnik w swojej karierze może być związany
kontraktem z wieloma drużynami. 


\subsubsection{Trener - Drużyna}
Podobnie jak w przypadku poprzedniego punktu, związek \emph{Trener - Drużyna} reprezentuje kontrakty trenerów z drużynami.
Jedna drużyna ma (zazwyczaj) jednego głównego trenera, jednak ma przeszłość z innymi szkoleniowcami. Z drugiej strony, jeden szkoleniowiec
może w swoim życiu być związany z kilkoma (rekordziści nawet z kilkunastoma) drużynami. 

\newpage

\section{Określenie atrybutów i ich dziedzin}
\vspace{1.5cm}
\subsubsection{Użytkownik}
\begin{table}[H]
	\begin{tabular}{|p{0.25\linewidth}|p{0.2\linewidth}|p{0.2\linewidth}|p{0.25\linewidth}|}
	\hline
	Nazwa atrybutu & Typ i dziedzina & Obowiązkowy? & Opis                                                           \\ \hline
	UżytkownikId   & liczbowy                            & TAK                              & Numer identyfikujący użytkownika                                                   \\ \hline
	Login          & napisowy                            & TAK                              & Login do systemu oraz nazwa pod którą użytkownik będzie widoczny w systemie         \\ \hline
	Hasło          & napisowy                            & TAK                              & Hash hasła, którym użytkownik będzie logował się do systemu                        \\ \hline
	Salt           & napisowy                            & TAK                              & Ciąg znaków zaburzający hasło, celem utrudnienia jego złamania                     \\ \hline
	Imię           & napisowy                            & TAK                              & Imię użytkownika, wymagane do identyfikacji i ewentualnego przekazania nagrody     \\ \hline
	Nazwisko       & napisowy                            & TAK                              & Nazwisko użytkownika, wymagane do identyfikacji i ewentualnego przekazania nagrody \\ \hline
	\end{tabular}
	\caption{Atrybuty encji Użytkownik}
\end{table}

\subsubsection{Mecz}
\begin{table}[H]
	\begin{tabular}{|p{0.25\linewidth}|p{0.2\linewidth}|p{0.2\linewidth}|p{0.25\linewidth}|}
	\hline
	Nazwa atrybutu & Typ i dziedzina & Obowiązkowy? & Opis                                                           \\ \hline
	MeczId   & liczbowy                            & TAK                              & Numer identyfikujący mecz                                                  \\ \hline
	DataRozpoczęcia          & datowy                            & TAK                              & Dzień i godzina rozpoczęcia meczu, do tego momentu można dodawać zakłady.       \\ \hline
	\end{tabular}
	\caption{Atrybuty encji Mecz}
\end{table}

\newpage

\subsubsection{Zakład}
\begin{table}[H]
	\begin{tabular}{|p{0.25\linewidth}|p{0.2\linewidth}|p{0.2\linewidth}|p{0.25\linewidth}|}
	\hline
	Nazwa atrybutu & Typ i dziedzina & Obowiązkowy? & Opis                                                           \\ \hline
	ZakładId   & liczbowy                            & TAK                              & Numer identyfikujący zakładu                                                   \\ \hline
	WynikGospodarzy         & liczbowy                           & TAK                              & Przewidywany wynik zdobyty przez drużynę gospodarzy         \\ \hline
	WynikGości          & liczbowy                            & TAK                              & Przewidywany wynik zdobyty przez drużynę gości                        \\ \hline
	DataZawarcia           & datowy                            & TAK                              & Data i godzina dodania/ostatniej edycji zakładu                 \\ \hline
	\end{tabular}
	\caption{Atrybuty encji Zakład}
\end{table}

\vspace{1cm}

\subsubsection{Wynik}
\begin{table}[H]
	\begin{tabular}{|p{0.25\linewidth}|p{0.2\linewidth}|p{0.2\linewidth}|p{0.25\linewidth}|}
	\hline
	Nazwa atrybutu & Typ i dziedzina & Obowiązkowy? & Opis                                                           \\ \hline
	WynikId   & liczbowy                            & TAK                              & Numer identyfikujący wynik                                                   \\ \hline
	WynikGospodarzy         & liczbowy                            & TAK                              & Faktyczny wynik zdobyty przez drużynę gospodarzy         \\ \hline
	WynikGości          & liczbowy                            & TAK                              & Faktyczny wynik zdobyty przez drużynę gości                        \\ \hline
	DataDodania           & datowy                            & TAK                              & Data i godzina dodania wyniku, potrzebna przy ewentualnej weryfikacji    \\ \hline
	\end{tabular}
	\caption{Atrybuty encji Wynik}
\end{table}

\vspace{1cm}

\subsubsection{Zakład Długoterminowy}
\begin{table}[H]
	\begin{tabular}{|p{0.25\linewidth}|p{0.2\linewidth}|p{0.2\linewidth}|p{0.25\linewidth}|}
	\hline
	Nazwa atrybutu & Typ i dziedzina & Obowiązkowy? & Opis                                                           \\ \hline
	ZakładDługotermi- nowyId   & liczbowy                            & TAK                              & Numer identyfikujący zakład długoterminowy                                                  \\ \hline
	DataZawarcia          & datowy                            & TAK                              & Data i godzina dodania/ostatniej edycji zakładu.       \\ \hline
	\end{tabular}
	\caption{Atrybuty encji Zakład Długoterminowy}
\end{table}

\newpage

\subsubsection{Turniej}
\begin{table}[H]
	\begin{tabular}{|p{0.25\linewidth}|p{0.2\linewidth}|p{0.2\linewidth}|p{0.25\linewidth}|}
	\hline
	Nazwa atrybutu & Typ i dziedzina & Obowiązkowy? & Opis                                                           \\ \hline
	TurniejId   & liczbowy                            & TAK                              & Numer identyfikujący turniej  \\ \hline
	Nazwa   & napisowy                            & TAK                              & Pełna nazwa turnieju  \\ \hline
	DataRozpoczęcia          & datowy                            & TAK                              & Data i godzina rozpoczęcia turnieju, tj. pierwszego spotkania w tym turnieju.       \\ \hline
	\end{tabular}
	\caption{Atrybuty encji Turniej}
\end{table}

\vspace{1cm}

\subsubsection{Drużyna}
\begin{table}[H]
	\begin{tabular}{|p{0.25\linewidth}|p{0.2\linewidth}|p{0.2\linewidth}|p{0.25\linewidth}|}
	\hline
	Nazwa atrybutu & Typ i dziedzina & Obowiązkowy? & Opis                                                           \\ \hline
	DrużynaId   & liczbowy                            & TAK                              & Numer identyfikujący drużynę                                                   \\ \hline
	Nazwa         & napisowy                           & TAK                              & Pełna nazwa drużyny         \\ \hline
	Miasto          & napisowy                            & TAK                              & Miasto z którego pochodzi drużyna                        \\ \hline
	Narodowość           & napisowy                            & TAK                              & Kraj z którego pochodzi drużyna                 \\ \hline
	\end{tabular}
	\caption{Atrybuty encji Drużyna}
\end{table}

\vspace{1cm}

\subsubsection{Stadion}
\begin{table}[H]
	\begin{tabular}{|p{0.25\linewidth}|p{0.2\linewidth}|p{0.2\linewidth}|p{0.25\linewidth}|}
	\hline
	Nazwa atrybutu & Typ i dziedzina & Obowiązkowy? & Opis                                                           \\ \hline
	StadionId   & liczbowy                            & TAK                              & Numer identyfikujący stadion                                                   \\ \hline
	Nazwa         & napisowy                           & TAK                              & Pełna nazwa areny         \\ \hline
	Pojemność 	   & liczbowy							& TAK								& Pojemność widowni \\  \hline
	Miasto          & napisowy                            & TAK                              & Miasto w którym znajduje się stadion                       \\ \hline
	Narodowość           & napisowy                            & TAK                              & Nazwa ulicy przy której znajduje się stadion   \\ \hline
	\end{tabular}
	\caption{Atrybuty encji Stadion}
\end{table}

\newpage

\subsubsection{Trener}
\begin{table}[H]
	\begin{tabular}{|p{0.25\linewidth}|p{0.2\linewidth}|p{0.2\linewidth}|p{0.25\linewidth}|}
	\hline
	Nazwa atrybutu & Typ i dziedzina & Obowiązkowy? & Opis                                                           \\ \hline
	TrenerId   & liczbowy                            & TAK                              & Numer identyfikujący trenera                                                   \\ \hline
	Imię         & napisowy                           & TAK                              & Imię/imiona trenera        \\ \hline
	Nazwisko	   & liczbowy							& TAK								& Nazwisko trenera \\  \hline
	DataUrodzenia          & datowy                           & TAK                              & Data urodzenia trenera              \\ \hline
	Narodowość           & napisowy                            & TAK                              & Kraj pochodzenia   \\ \hline
	\end{tabular}
	\caption{Atrybuty encji Trener}
\end{table}

\vspace{1cm}

\subsubsection{Sędzia}
\begin{table}[H]
	\begin{tabular}{|p{0.25\linewidth}|p{0.2\linewidth}|p{0.2\linewidth}|p{0.25\linewidth}|}
	\hline
	Nazwa atrybutu & Typ i dziedzina & Obowiązkowy? & Opis                                                           \\ \hline
	SędziaId   & liczbowy                            & TAK                              & Numer identyfikujący sędziego                                                   \\ \hline
	Imię         & napisowy                           & TAK                              & Imię/imiona sędziego        \\ \hline
	Nazwisko	   & liczbowy							& TAK								& Nazwisko sędziego \\  \hline
	DataUrodzenia          & datowy                           & TAK                              & Data urodzenia sędziego              \\ \hline
	Narodowość           & napisowy                            & TAK                              & Kraj pochodzenia   \\ \hline
	\end{tabular}
	\caption{Atrybuty encji Sędzia}
\end{table}

\section{Dodatkowe reguły integralnościowe}

Należy zapewnić aby w systemie nie doszło do sytuacji że zwycięzcą danego turnieju jest drużyna która nie jest jego uczestnikiem.
Zgodnie podjęliśmy decyzję projektową że to zabezpieczenie zostanie zrealizowane po stronie aplikacji a nie po stronie bazy danych.

\section{Klucze kandydujące i główne}



\section{Schemat ER na poziomie konceptualnym}

\section{Problem pułapek szczelinowych i wachlarzowych – analiza i przykłady}



\chapter{Model logiczny}

\section{Charakterystyka modelu relacyjnego}

\section{Usunięcie właściwości niekompatybilnych z modelem relacyjnym}

\section{Proces normalizacji}

\subsection{Pierwsza postać normalna}
\subsection{Druga postać normalna}
\subsection{Trzecia postać normalna}


\section{Schemat ER na poziomie modelu logicznego}

\section{Więzy integralności}

\section{Proces denormalizacji}

\chapter{Faza fizyczna}

\section{Projekt transakcji i weryfikacja ich wykonalności}

\section{Strojenie bazy danych – dobór indeksów}

\section{Skrypt SQL zakładający bazę danych}

\section{Przykłady zapytań i poleceń SQL odnoszących się do bazy danych}

\begin{thebibliography}{9}

	

	\bibitem{Wrembel1}
	  Robert Wrembel,
	  \emph{Wykład z przedmiotu Bazy Danych. Wykład 3: Modelowanie danych, Model związków-encji}.,
	  Poznań,
	  2006.

	
	\bibitem{Kowalczyk1}
	  Marcin Kowalczyk,
	  \emph{Wykład z przedmiotu Wprowadzanie do Baz Danych}.,
	  Warszawa,
	  20018.
	
	\end{thebibliography}

\end{document}